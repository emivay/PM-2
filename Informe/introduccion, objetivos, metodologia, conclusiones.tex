 \documentclass[12pt, titlepage]{article}
 \usepackage{amsmath, empheq}
 \usepackage{amssymb}
 \usepackage{geometry}
 \geometry{
 	a4paper,
 	total={170mm,257mm},
 	left=20mm,
 	top=20mm,
 }
\usepackage[utf8]{inputenc}
\usepackage{graphicx}
\usepackage[font=small,labelfont=bf]{caption} % Required for specifying captions to tables and figures

\begin{document}


\section{Introducción}
\section{Objetivos}
Los objetivos que se plantean para la práctica són:
\begin{enumerate}
\item	Comprender el fundamento teórico de los algoritmos del Símplex primal y dual.
\item	Implementar el algoritmo del Símplex primal
\end{enumerate}
\section{Metologia}
Los objetivos anteriormente propuestos se alcanzarán con la siguiente metodología:
\begin{enumerate}
\item	Se realizará un estudio exhaustivo de los fundamentos de la programación lineal y del algoritmo del Símplex. Esta etapa, en su mayor parte, se hará durante la época previa a los exámenes parciales.
\item	Se implementará el algoritmo del Símplex primal en dos lenguajes de programación: C++ y MATLAB.
\end{enumerate}
\section{Conclusiones}
Las conclusiones que hemos podido extraer tras la realización de la práctica son:
\begin{enumerate}
\item	asdasdas
\item	sdfsafdsfds
\end{enumerate}
Otras conclusiones a las que hemos llegado además de las de los objetivos planteados son:
\begin{enumerate}
\item	asdasdsa
\item	dasdas
\end{enumerate}
	
\end{document}