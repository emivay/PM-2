 \documentclass[12pt, titlepage]{article}
 \usepackage{amsmath, empheq}
 \usepackage{amssymb}
 \usepackage{geometry}
 \geometry{
 	a4paper,
 	total={170mm,257mm},
 	left=20mm,
 	top=20mm,
 }
\usepackage[utf8]{inputenc}
\usepackage{graphicx}
\usepackage[font=small,labelfont=bf]{caption} % Required for specifying captions to tables and figures

\begin{document}


\section{Introducción}
\section{Objetivos}
Los objetivos que se plantean para la práctica són:
\begin{enumerate}
\item	Comprender el fundamento teórico del algoritmos del Símplex primal y su aplicación a la resolución de problemas de programación lineal.
\item	Implementar el algoritmo del Símplex primal.
\end{enumerate}
\section{Metologia}
Los objetivos anteriormente propuestos se alcanzarán con la siguiente metodología:
\begin{enumerate}
\item	Se realizará un estudio exhaustivo de los fundamentos de la programación lineal y del algoritmo del Símplex. En su mayor parte, este objetivo se realizará durante la época de estudio de exámenes parciales.
\item	Se implementará el algoritmo del Símplex primal en dos lenguajes de programación, C++ y MATLAB, y se compararán las implementaciones.
\end{enumerate}
\section{Conclusiones}
Las conclusiones que hemos podido extraer tras la realización de la práctica son:
\begin{enumerate}
\item	Hemos aprendido los fundamentos teóricos del Símplex primal y sabemos aplicarlo para resolver un problema de programación lineal. Así mismo, hemos visto que la manera en que se nos ha enseñado y hemos estudiado el algoritmo es más sencilla y bella que el método del tablón del Símplex, 
\item	Hemos conseguido desarrollar el Símplex en MATLAB y en C++, y hemos observado diferencias significativas entre ambas implementaciones. En la primera, el algoritmo funciona de manera correcta. El código resultante tiene un sintaxis semejante al lenguaje matemático, por tanto es fácilmente comprensible y la extensión del programa es reducida. Por el contrario, la implementación en C++ ha sido más costosa y únicamente funciona en determinados problemas. Concluimos, por lo tanto, que es recomendable implementarlo en un lenguaje pensado específicamente para el trabajo de objetos matemáticos, como puede ser MATLAB u Octave.
\end{enumerate}
	
\end{document}